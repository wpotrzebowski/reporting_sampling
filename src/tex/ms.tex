% Define document class
\documentclass[reprint,superscriptaddress,aps,amsmath,linenumbers]{revtex4-2}

\usepackage{showyourwork}
\usepackage{graphicx}
\usepackage[separate-uncertainty=true]{siunitx}
\sisetup{
    list-units = brackets,
    range-units = brackets,
    range-phrase = -,
    list-pair-separator = {, },
    list-separator = {, },
    list-final-separator = {, }
    }

% Begin!
\begin{document}

% Title
\title{Advice on describing Bayesian analysis of neutron and X-ray reflectometry}
\thanks{This work was developed as a part of the Open Reflectometry Standards Organisation Workshop in 2022 with contributions from all authors as part of a round-table discussion.}

% Author list
\author{Andrew R. McCluskey}
  \email{andrew.mccluskey@ess.eu}
  \affiliation{European Spallation Source ERIC, P.O. Box 176, SE-221 00, Lund, SE}
\author{Andrew J. Caruana}
  \affiliation{ISIS-Neutron and Muon Source, Rutherford Appleton Laboratory, Didcot, Oxon OX11 0QX, GB}
\author{Christy J. Kinane}
  \affiliation{ISIS-Neutron and Muon Source, Rutherford Appleton Laboratory, Didcot, Oxon OX11 0QX, GB}
\author{Alexander J. Armstrong}
  \affiliation{ISIS-Neutron and Muon Source, Rutherford Appleton Laboratory, Didcot, Oxon OX11 0QX, GB}
\author{Tom Arnold}
  \affiliation{European Spallation Source ERIC, P.O. Box 176, SE-221 00, Lund, SE}
\author{Joshaniel F. K. Cooper}
  \affiliation{ISIS-Neutron and Muon Source, Rutherford Appleton Laboratory, Didcot, Oxon OX11 0QX, GB}
\author{David L. Cortie}
  \affiliation{Australian Nuclear Science and Technology Organisation, Lucas Heights, New South Wales, AU}  
\author{Alessandro Greco}
  \affiliation{Institute of Applied Physics, University of T\"{u}bingen, Auf der Morgenstelle 10, 72076 T\"{u}bingen, DE}
\author{Arwel V. Hughes}
  \affiliation{ISIS-Neutron and Muon Source, Rutherford Appleton Laboratory, Didcot, Oxon OX11 0QX, GB}
\author{Gaetano Mangiapia}
  \affiliation{German Engineering Material Science at Heinz Maier-Leibnitz Zentrum, Helmholtz-Zentrum Hereon, Lichtenbergstraße 1, 85748 Garching, DE}  
\author{Brian B. Maranville}
  \affiliation{Center for Neutron Research, National Institute of Standards and Technology, 100 Bureau Drive, Gaithersburg, Maryland 20899, US}
\author{Jean-Fran\c{c}ois Moulin}
  \affiliation{German Engineering Material Science at Heinz Maier-Leibnitz Zentrum, Helmholtz-Zentrum Hereon, Lichtenbergstraße 1, 85748 Garching, DE}
\author{Andrew R. J. Nelson}
  \affiliation{Australian Nuclear Science and Technology Organisation, Lucas Heights, New South Wales, AU}  
\author{Mariano Andr\'{e}s Paulin}
  \affiliation{Laboratoire L\'{e}on Brillouin. UMR12 CEA-CNRS, B\^{a}t. 563 CEA Saclay. 91191 Gif sur Yvette Cedex, FR}
\author{Wojciech Potrzebowski}
  \affiliation{European Spallation Source ERIC, P.O. Box 176, SE-221 00, Lund, SE}
\author{Mark L. Schlossman}
  \affiliation{Department of Physics, University of Illinois at Chicago, Chicago, Illinois 60607, US}
\author{Jochen Stahn}
  \affiliation{Paul Scherrer Institute, 5232, Villigen, CH}
\author{Vladimir Starostin}
  \affiliation{Institute of Applied Physics, University of T\"{u}bingen, Auf der Morgenstelle 10, 72076 T\"{u}bingen, DE}
  
\collaboration{Open Reflectometry Standards Organisation}

% Abstract with filler text
\begin{abstract}
    Lorem ipsum dolor sit amet, consectetuer adipiscing elit. 
    Ut purus elit, vestibulum ut, placerat ac, adipiscing vitae, felis. 
    Curabitur dictum gravida mauris, consectetuer id, vulputate a, magna. 
    Donec vehicula augue eu neque, morbi tristique senectus et netus et. 
    Mauris ut leo, cras viverra metus rhoncus sem, nulla et lectus vestibulum. 
    Phasellus eu tellus sit amet tortor gravida placerat. 
    Integer sapien est, iaculis in, pretium quis, viverra ac, nunc. 
    Praesent eget sem vel leo ultrices bibendum. 
    Aenean faucibus, morbi dolor nulla, malesuada eu, pulvinar at, mollis ac. 
    Curabitur auctor semper nulla donec varius orci eget risus. 
    Duis nibh mi, congue eu, accumsan eleifend, sagittis quis, diam. 
    Duis eget orci sit amet orci dignissim rutrum.
\end{abstract}

\maketitle

% Main body with filler text
\section{Introduction}
\label{sec:intro}

Neutron and X-ray reflectometry are powerful tools to probe the interfacial structure of materials \cite{lovell_analysis_1999}.
However, as a result of the ``phase problem'', the analysis of these techniuqes is ill-posed in nature, as there are multiple possible solutions \cite{majkrzak_exact_1995}.
This has led to the use of Bayesian analysis, where some prior understanding of the system is used to aid our understanding of some reflectivity profile \cite{sivia_analysis_1991,geoghegan_experimental_1996,sivia_bayesian_1998}.
Recently, developments in the availability of computer software for reflectometry analysis that include Bayesian functionality, such as Refl1d, refnx, anaklasis, and RasCAL \cite{kienzle_refl1d_2021,nelson_refnx_2019,koutsioubas_anaklasis_2021,hughes_rascal_2019}, which implement methods from bumps, emcee, and dynesty \cite{kienzle_bumps_2021,foremanmackey_emcee_2019,speagle_dynesty_2020}, have led to an increase in the utilisation of Bayesian methods by the reflectometry community \cite{mccluskey_bayesian_2019,mccluskey_general_2020}.

The discussion in this work will focus on best practice for reporting the results from Bayesian- and sampling-based analysis of neutron and X-ray reflectivity data. 
This work will not introduce Bayesian or sampling methods for neutron and X-ray reflectometry analysis, for those unfamiliar with these techniques, we suggest the work of Sivia and co-workers \cite{sivia_bayesian_1998,sivia_data_2006} and more recent work focusing on reflectometry analysis \cite{hughes_physical_2019,mccluskey_general_2020,nelson_refnx_2019,aboljadayel_determining_2021}. 
We hope that this work will inform best practices in data sharing from reflectometry analysis and inspire software developers to enable these to be accessed easily by the user. 

Reflectometry analysis can be described, in the most simplistic terms, as a comparison and refinement of a model based on some parameters, $\mathbf{x}$, to reproduce some reflectivity data set, $\mathbf{D}$. 
This refinement process involves comparing the model to the data and calculating some goodness-of-fit parameter or likelihood, $p(\mathbf{D} | \mathbf{x})$, and modifying the model to either minimise or maximise this, as appropriate. 
A commonly used goodness-of-fit parameter is the $\chi^2$ parameter \cite{nelson_refnx_2019} which is found as, 
%
\begin{equation}
    \chi^2 = \sum_{q=q_{\text{min}}}^{q_{\text{max}}}{\bigg[\frac{(R(q) - R(q)_{\text{m}})}{\sigma_R(q)}\bigg]^2}, 
\end{equation}
%
where, $R(q)$ is the experimental measured reflectivity, and $\sigma_R(q)$ is the associated uncertainty, at a given $q$ and $R(q)_{\text{m}}$ is the calculated reflectivity for the model at the same $q$.
The likelihood is related to the $\chi^2$ parameter, however, a larger likelihood is indicative of better agreement, 
%
\begin{equation}
    \ln[p(\mathbf{D} | \mathbf{x})] = -\frac{1}{2} \bigg(\chi^2 + \sum_{q=q_{\text{min}}}^{q_{\text{max}}}\ln{\big[2\pi\sigma_R(q)^2\big]}\bigg).
\end{equation}
%
The input for this refinement process is the model and some initial parameter values, which may be an absolute value or some parameter range, depending on the refinement algorithm. 
The output is a set of values for $\mathbf{x}$, potentially with associated error bars, where these are present they typically describe a standard deviation from the mean of a Gaussian probability distribution. 
This process implicitly assumes that the data is completely reduced, accounting for all experimental parameters, uncertainties are accurately described and the model can accurately describe the data. 

The input required depends on a minimisation algorithm being used, with some algorithms requiring a single starting guess (such as traditional Newtonian methods) and others taking a range of potential values (more common in stochastic approaches, like differential evolution). 
The nature of these inputs define the results of the analysis, therefore it is of the utmost importance that these are shared as part of a publication describing the work. 
Furthermore, often the minimisation is performed with bounds in place, defining that the parameter values will lie within a given range. 
This range can be throught of as a having a prior probability distribution, where values of $x$ outside of this range have a probability of \num{0}. 
Even when a non-Bayesian approach is used in the analysis (i.e. Bayes theorem is not utilised), the result where bounds are set would be analogous to a Bayesian analysis with a uniform prior probability. 



\section*{CRediT author statement}
\label{sec:credit}

A.R.M: Conceptualization, Methodology, Resources, Writing - original draft, Writing - review \& editing, Visualisation, Project administration.
A.J.C \& C.J.K.: Conceptualization, Writing - review \& editing.
Other authors: Writing - review \& editing.

\begin{acknowledgments}
    We acknowledge.
\end{acknowledgments}

\bibliographystyle{naturemag}
\bibliography{bib}

\end{document}
